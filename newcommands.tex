%------------------------------------ Set new commands ---------

%%%%%% General %%%%%%%%%%%%%%%%%%%%
\newcommand{\ve}[1] %make vector notation
{{\bf #1}}

\newcommand{\unit}[1] %Units
{\, \mathrm{#1}}

\newcommand{\Bl} % Big left
{{\Big ( }}
\newcommand{\bl} % Big left
{{\big ( }}

\newcommand{\Bh} % Big left
{{\Big ) }}
\newcommand{\bh} % Big left
{{\big ) }} 


%%%%%% Theorems %%%%%%%%%%%%%%%%%%
\usepackage{amsmath}               
  {
      \theoremstyle{plain}
      \newtheorem{assumption}{Assumption}
  }
\usepackage{amsmath}               
  {
      \theoremstyle{plain}
      \newtheorem{theorem}{Theorem}
  }
  
  \usepackage{amsmath}               
  {
  	\theoremstyle{plain}
  	\newtheorem{lemma}{Lemma}
  }
\newcommand{\refas}[1]
{A(\ref{#1})}

%%%%%% Mathematics %%%%%%%%%%%%%%%%%%
\newcommand{\abs}[1]{\lvert#1\rvert} %Abs
\newcommand{\norm}[1]{\lVert#1\rVert} %Norm

%Case 
\newcommand{\cif}  %If in cases
{\text{ if }}

\newcommand{\celse}  %If in cases
{\text{ else}}

%Partial derivatives
\newcommand{\dxdy}[2] %make a dx/dy with d as normally done
{\frac{\text{d}#1}{\text{d}#2}}

\newcommand{\ddxdy}[2] %make a dx/dy with d as normally done
{\frac{\partial \, #1}{\partial#2}}




\newcommand{\dx}[1] %make a dx
{{\Delta#1}}

\newcommand{\ud}{\,\mathrm{d}} %make nice dx in integral

\newcommand{\dder}[2] %discrete derivate notation
{ #2^* }
%Other
\newcommand{\nint}[1]
{\text{nint} \, #1}
\newcommand{\bsigma}
{\bold{\sigma}}
\newcommand{\lse}[1]
{\mathrm{lse} \, #1}

\DeclareMathOperator*{\argmax}{arg\,max}

%\newcommand{\unit}[1] %make vector notation
%{\, \mathrm{#1}}

%%%%% References to equations, tables%%%%
\newcommand{\refeq}[1]
{(\ref{#1})}

%%%%% Notations for estimation%%%%%%%
\newcommand{\Ll} % Likelihood notation
{\mathscr{L}}
\newcommand{\LL} % Log-likelihood notation
{\mathscr{LL}}
\newcommand{\pp} % Probability
{\mathscr{P}}
\newcommand{\obs} % Observation set
{\mathscr{O}}

%%%% Dynp %%%%%%%%%%%
% General dynp notations
\newcommand{\EV} % Expected utility notation
{\eutil}
% Notations for state spaces in dynp
\newcommand{\Cc} %Write choice set
{\mathbb{C}}
\newcommand{\Ccsamp}
{\tilde{\Cc}}
\newcommand{\Uu} %write U set
{\mathbb{U}}
\newcommand{\xx} %write x set
{\mathbb{X}}
\newcommand{\uu} % write u set
{\mathbb{U}}
\newcommand{\rr} % write R set
{\mathbb{R}}
\newcommand{\Agood} % write action-sequence set
{\mathscr{A}}
\newcommand{\seqspace} % Write statespace
{\mathscr{S}}
\newcommand{\eutil} % Expected utility notation
{\ensuremath{EV}}
\newcommand{\param}[1]
{{\ensuremath{\theta_{\text{#1}}}\xspace}}
\newcommand{\laeutil} % Expected utility notation
{\ensuremath{\overline{\eutil}}\xspace}
\newcommand{\aeutil} % Expected utility notation
{\ensuremath{\widetilde{\eutil}}\xspace}


\newcommand{\med}[1]
{ \bar{#1}}

\newcommand{\TB}{T_{\beta}}
%FOR TIKZ%%%%%%%%%%%%%%%%%%%%%%%%%%%%%%%%%%%%
\usepackage{tikz}
\usetikzlibrary{decorations.markings}
\usetikzlibrary{shapes.geometric}

\pgfdeclarelayer{edgelayer}
\pgfdeclarelayer{nodelayer}
\pgfsetlayers{edgelayer,nodelayer,main}

\tikzstyle{none}=[inner sep=0pt]
\tikzstyle{rn}=[circle,fill=blue!40,draw=black,line width=0.8 pt]
\tikzstyle{gn}=[circle,fill=yellow!40,draw=black,line width=0.8 pt]
\tikzstyle{yn}=[circle,fill=red!40,draw=black,line width=0.8 pt]
\tikzstyle{rnn}=[circle,fill=blue!40,draw=black,line width=0.3 pt,inner sep=0pt, minimum size=0.10cm]
\tikzstyle{gnn}=[circle,fill=yellow!40,draw=black,line width=0.3pt,inner sep=0pt, minimum size=0.10cm]
\tikzstyle{ynn}=[circle,fill=red!40,draw=black,line width=0.3 pt,inner sep=0pt, minimum size=0.10cm]
\tikzstyle{enn}=[circle,fill=black!40,draw=black,line width=0.3 pt,inner sep=0pt, minimum size=0.10cm]
\tikzstyle{wn}=[circle,fill=white,draw=black,line width=0.8 pt]
\tikzstyle{simple}=[->,dashed,draw=black!80,line width=0.400]
\tikzstyle{arrow}=[->,draw=black,line width=1.000]
\tikzstyle{tick}=[-,draw=black, line width = 0.6]
\tikzstyle{simpleSmall}=[->,dashed,draw=black!80,line width=0.100]
\tikzstyle{arrowSmall}=[->,draw=black!80,line width=0.2000]
\tikzstyle{arrowMedium}=[->,draw=black,line width=0.4000]
\tikzstyle{arrowTiny}=[->,draw=black!30,line width=0.1000]
\tikzstyle{tickSmall}=[-,draw=black, line width = 0.2]
\tikzstyle{block} = [draw=white, fill=white!20, rectangle,text centered,opacity=0,text opacity = 1,font=\scriptsize ]
\tikzstyle{onlytext}=[block,text width=1.5cm]
