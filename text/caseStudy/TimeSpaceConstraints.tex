Time space constraints define when and where an individual can participate in different activities and thereby impose a structure on the day. Time-space constraints can be of the type ``I have to be at work by $7\unit{a.m.}$'', and both explicitly determine where an individual will be at $7\unit{a.m.}$ (at work) and implicitly influence where they can be at $6:50\unit{a.m.}$ (not more than $10\unit{minutes}$ away from work with available modes of transport). To check that a specific trip is possible, one must look multiple future trips into the future to ensure that all time-space constraints can be satisfied if that specific trip is carried out. Finding feasible activity schedules in a dynamic discrete choice model is trivial since expected value function $\eutil = -\infty$ in any explicitly or implicitly infeasible state, as by definition there are no actions leading from such a state to another state with $\eutil \neq -\infty$. Actions that are implicitly infeasible due to time-space constraints will therefore have zero probability.%How is time-space constraints usually modelled? Does usually not influence choices on higher levels.

Some activities are time constrained. Time constraints on when activities can be started or when they must be completed can easily be included by restricting the choice set at times that do not meet these constraints. Location constraints, i.e., constraints specifying where different activities can take place, are treated in the same way.

People can have fixed or flexible working hours. People with fixed working hours must arrive at work when the workday start and leave when the workday ends. %Specify how this is done: +-10minutes is ok.
People with flexible working hours can choose to arrive between $6\unit{a.m.}$ and $10\unit{a.m.}$, but the length of a working day is still fixed. The individual specifications on working hour type, working length, start and end hours must be provided from elsewhere. Children can be dropped off between $6:30\unit{a.m.}$ and $12:00\unit{a.m.}$. Pick up trips must be completed between $12\unit{a.m.}$ and $6:30\unit{p.m.}$
All individuals must start and end their days at home. There is no need to restrict the state space in the start of the day. Such restrictions are ensured by the choice of the initial state used when, e.g., simulating day paths.


Picking up and dropping of children at school as well as going to work are considered mandatory activities with fixed location and time constraints. These three activities further have an internal order: dropping of children is done before going to work which must be done before picking up the children again. To model this order of activities, we introduce the errand indicator $E$. When $E=0$, only dropping of children is possible. After having finished a drop-off activity, $E$ increases by one and the only available activity in the group is work. Enforcing that all activities are finished during the day is done by restricting $E$ in the end of the day, and time-constraints are treated as above.

More generally, a constraint could impose that some activity or a group of activities must be conducted a number of times $N$ during a day. This can be modeled by introducing an errand indicator state variable, say $Q$, for each such group of mandatory activities and setting the expected value function to $-\infty$ whenever $Q\neq N$ in the end of the day. Whenever an activity in the group is started, $Q$ is increased by one. If the day is started in a state with $Q=0$, all feasible activity schedules will do activities in the group exactly $N$ times. Introducing an extra state variable is not without costs. The number of states will increase linearly with $N$ and the number of actions in each state will not decrease substantially, so the computation time will increase almost linearly with $N$ in each basic activity constraint. If there are multiple groups of mandatory activities where the activities in a group $i$ must be conducted $N_i$ times, the number of states will increase with a factor $\prod_i (N_i+1)$ times. 