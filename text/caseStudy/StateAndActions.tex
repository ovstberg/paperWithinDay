A state should include the information needed to determine available actions and utility of these actions. Here a state $x$ consists of:
\begin{description}[style=multiline,leftmargin=4cm,font=\normalfont]
\item[{Time $t \in [5\unit{am},11\unit{pm}] $}:] Continuous variable for time of day. A day starts at $5\unit{am}$ and ends at $11\unit{pm}$
\item[{Location $L \in [1,1240]$:}] Current location. One of 1240 zones in the region of Stockholm. 
\item[Activity $A$:] New activity, end activity, social, recreational, shop, home, work and escorting children are the alternative activity states. 

The activity must be included in the state since the individual can choose to continue with the same activity for yet another time-period, but have to travel (possibly within the zone) to change activity. The purpose of the new activity and end activity states will be discussed below.
\item[{Errand indicator $E \in [0,3]$:}] A state keeping track of the number of finished mandatory activities. The number of mandatory activities varies from 1 to 3 depending on the individual, as will be explained later. 
\item[Car dummy $\delta_{car} \in \{true,false\}$:] Dummy for car availability. An individual have to travel with car if $\delta_{car} = true$ and if out of home and cannot travel with car if $\delta_{car} = false$.
\end{description}


The set of actions $a$ that are available in a state $x$ for individual $n$ is denoted $A_n(x)$. The universal choice set consists of any combination of activity $A$, mode $M$ and location $L$:
\begin{description}[style=multiline,leftmargin=4cm,font=\normalfont]
\item[{Activity $A$}:] Activity for new action.
\item[{Location $L$}:] New location. 
\item[{Mode}:] Car, public transport, bike and walk are the modelled modes. When continuing with the same activity, the mode of the action is ``no-mode''.
\end{description}

