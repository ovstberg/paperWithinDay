
Consider an individual $n$ who has been observed to choose a day-path starting from state $x_0$ consisting of a sequence of decisions  $\ve{a} = (a_0,...,a_{T-1})$ and thus traversing the states $\ve{x} = (x_1,...,x_{T})$. The likelihood for this sequence of decisions is, according to \eqref{eq:MNLv2} given by:
 \begin{equation}\label{eq:pPath0}
 \begin{aligned}
 P(\ve{a}|x_0) & = \prod_{i=0}^{T}P(a_i|x_i) \\
 & = \prod_{i=0}^{T} e^{u(a_i,x_i) + \eutil(x_{i+1})-\eutil(x_i)}
 \end{aligned}
 \end{equation}
The standard method for estimating dynamic discrete choice models is the Nested Fixed Point Method (NFXP) \citep{RustML88}. This involves first calculating $\eutil$ and its gradients in each state of the network and then directly use \eqref{eq:pPath0} for estimation. Although possible to apply on the proposed model, the method would be extremely time consuming given the discussion on computation time in Section \ref{seq:computationTime}. If estimation would rely on calculating the value function and gradient of the value function in each state in each iteration, the computation time would likely be $0.4\cdot N_{\text{variables}} \cdot t$ per observation per iteration (as the gradient must be calculated for each variable, requiring at least the additional sum of the gradient of $u + EV$ for each variable). With $10\units{s/observation}$ to calculate $\eutil$, $70$ variables, $3\zdel 300$ observations and $100$ iterations before convergence, this would require $10 \cdot 0.4 \cdot  70 \cdot 3\zdel 300 \cdot 100 \units{s} \approx 1\zdel 000 \units{days}$ to estimate using a single core. 

Methods used to speed up estimation of dynamic discrete choice models, e.g., the method proposed in \citet{aguirregabiria2002swapping}, reduces the burden of value iterations when calculating the value function. As no iteration is needed to evaluate the value function in this model, this would likely not help. Some approximative method is therefore needed.

\subsection{Sampling of alternatives}
In the context of route choice modeling, \citet{fosgerau2013} showed that an MNL over routes in a directed network can be expressed by \eqref{eq:MNLv2} and estimated using the NFXP. From \eqref{eq:pPath0}, observe that: 
\begin{equation}\label{eq:pPath}
\begin{aligned}
P(\ve{a}|x_0) & = \prod_{i=0}^{T} e^{u(a_i,x_i) + \eutil(x_{i+1})-\eutil(x_i)} \\
& = e^{u(\ve{a},x_0) + \eutil(x_{T+1})-\eutil(x_0)}
\end{aligned}
\end{equation}
where $u(\ve{a},x_0) = \sum_{i=0}^{T-1} u(a_i,x_i)$. Since $\eutil(x_{T})$ and $\eutil(x_0)$ are the same for all alternative action sequences starting in $x_0$, the probability for each alternative is proportional to $e^{u(\ve{a},x_0) }$. If $\Agood(x_0)$ is the set of action sequences that, starting from $x_0$, satisfies all space-time constraints, then:
\begin{equation} \label{eq:MNLpath}
P(\ve{a}|x_0) = \frac{e^{u(\ve{a},x_0)}}{\sum\limits_{\ve{a'} \in \Agood(x_0)}e^{u(\ve{a'},x_0)}}.
\end{equation}
Recursive Logit (RL) models has recently been developed extending the MNL case discussed in \citet{fosgerau2013} to cover Nested Logit \citep{mai2015}, MEV \citep{mai2016method} and Mixed Logit specifications \citep{mai2016decomposition}. They have also been applied in a number of scenarios, e.g., in \cite{zimmermann2017bike}  application to route choice for bikes, possibly with the largest network so far in a RL model. The number of links in these models are between 7\zdel 000-40\zdel 000. The model presented here is thus around $2\zdel 000$ times bigger, so although the models are very similar the estimation techniques used for RL-models are to feasible here. 

Here, the equivalence between a RL-model in \eqref{eq:pPath} and an MNL over routes in \eqref{eq:MNLpath} is utilized to allow estimation using sampling of alternatives. 
It is a general property of MNL models that estimating over a subset of alternatives gives consistent estimates if a correction term is added to the utility function \citep{mcfadden78}. The estimates are, however, not efficient and how the choice sets are constructed will determine the efficiency loss. Since the number of alternatives is immense, is is important to use a smart way of sampling alternatives that somewhat resembles the model in order to obtain good estimates. As it is trivial to simulate alternative once the value function has been evaluated, a choice set can be constructed using the described model with an initial guess of the parameters.

Estimation using sampling of alternatives involves sampling a choice set $\Ccsamp_n \subset \Cc_n$ and estimating using the conditional choice probability $P_n(\ve{a}_n|\Ccsamp_n)$ instead of the $P_n(\ve{a}_n|\Cc_n)$. A maximum likelihood estimation on a choice set $\Ccsamp$ gives consistent estimates if the correction term $\log(\bar{q}_n(\Ccsamp_n|j))$ is added to each alternative and $\bar{q}_n(\Ccsamp_n|j)$ satisfies the positive conditioning property, i.e., that if $j\in \Ccsamp_n$ and $\bar{q}_n(\Ccsamp_n|i)>0$ for some $i$, then $\bar{q}_n(\Ccsamp_n|j)>0$. This holds if $\Ccsamp_n$ is sampled from the universal choice set $\Cc_n$ and all alternatives in $\Cc_n$ have a non-zero probability of being sampled.

Let $N$ observations form the set of observations $\obs_N$. The log-likelihood function for $\obs_N$ based on the conditional likelihoods becomes:
\begin{equation} \label{eq:cond_like}
\bar{\LL}(\obs_N;\theta) = \sum_{n=1}^N \log \Bl \frac{e^{u(\ve{a}_n) + \log(q_n(\Ccsamp_n|j))}}{\sum\limits_{\ve{a}^* \in \Ccsamp_n} e^{u(\ve{a}^*) + q_n(\Ccsamp_n|\ve{a}^*)}}\Bh
\end{equation}
If all alternatives in $\Cc_n$ have equal probability of being sampled to the choice set $\Ccsamp$, the correction term $q_n(\Ccsamp_n|j)$ will also be the same for all alternatives and therefore cancel out from the likelihood function. However, if some other sampling protocol is used the probability must be calculated. The sampling protocol use here is the same that \citet{frejinger09} used to estimate an MNL model over the choice of routes in a traffic network. The sampling protocol consists of drawing $R$ alternatives with replacement from the choice set $\Cc_n$ consisting of $J_n$ alternatives, and then adding the observed choice to the choice set. The outcome of such a protocol is $({k}_{n1},{k}_{n2},\dots,{k}_{nJ})$ where ${k}_{nj}$ is the number of times alternative $j$ appears in the choice set, so that $\sum_{j=1}^J {k}_{nj} = R+1$, since the observed alternative $j$ is added once extra to the choice set. Let $q_n(i)$ denote the probability that alternative $j \in \Cc_n$ is sampled. The correction term can then be derived to: $q_n(\Ccsamp_n|j) = K_{\Ccsamp_n}\frac{k_{nj}}{q_n(j)}$. The constant $K_{\Ccsamp_n}$ will cancel out from the likelihood function to give:
\begin{equation} \label{eq:cond_like_corr}
\bar{\LL}(\obs_N;\theta) = \sum_{n=1}^N \log \Bl \frac{e^{u(\ve{a}_n) + \log(\frac{k_{n\ve{a}_n}}{q_n(\ve{a}_n)})}}{\sum\limits_{\ve{a}^* \in \Cc_n} e^{u(\ve{a}^*) + \log(\frac{k_{n\ve{a}^*_n}}{q_n(\ve{a}^*_n)})}}\Bh
\end{equation}
 
As previously mentioned, a choice set is sampled using \refeq{eq:MNL} with a single set of parameters. They were derived by starting from a simple specification of the model, only involving time and cost parameters, and then manually their values until travel times, mode choices and activity episodes were in line with the observations.

\subsection{Correlation between alternatives}
It is common practice in route choice modelling to add a size attribute to each link to take correlation among paths that overlap into account, e.g., using Path-Size Logit \citep{BenAkivaBier99}. For their link-based route choice model, \citet{fosgerau11} obtains a size coefficient by calculating \eutil\, in each link using some pre-specified parameters and adding that to the link-utility. In the activity-scheduling model presented here, it is not as easy to define the overlapping of paths, as the network is dynamic. If two paths are identical besides that the start time for all activities in one path is $10\unit{min}$ after the start time in the other path, there can be practically no overlapping as defined by the Path-Size Logit although the two paths would be very similar. How to address this issue in a dynamic network and in the activity-scheduling framework is therefore an open question.

In trip-generation models, it is common to have nests for mode choice, location choice and activity choice, as in, e.g., \citet{Bowman01}. It would be possible to introduce different scales for the error term when solving \refeq{eq:EV} and obtaining choice probabilities in \refeq{eq:MNL} where the scale (which is one here) would be state dependent. This is done in the Nested Recursive Logit model described in \citet{mai2015}. However, the probability of a path would then not reduce to  \refeq{eq:pPath} and sampling of alternative sequences would not be possible to use for estimation.
\citet{guevara2013MEV} recently showed that Multivariate Extreme Value (MEV) models such as Nested Logit can be estimated using sampling of alternatives. A Nested RL-model is however not the same as an MEV model, so the transferability of the result is uncertain. If nests are introduced within the network, as in \citet{mai2015}, it would further require that the value function was approximated in all states, so the computational benefit might not be enough. An alternative would be to introduce nests over paths, for example nesting alternatives that include specific activities or modes. This would move the model in the direction of \citet{Bowman01}. The computation time is however likely to grow linearly with the number of nests. Given that the model already is time demanding, creating nests for all combinations of modes and activities would not be computationally feasible.

 Another issue is the correlation in preferences over time. Individuals' variances in preferences for, e.g., mode or activities are likely to be consistent over time and therefore to some extent be the same throughout the day. Including nests on a trip level would not capture this correlation. A possible solution would be to introduce mixed parameters for, e.g., activities and modes, that would be the same for each individual for the full day. Our estimation approach is based on sampling of alternatives and recent research by \citet{Guevara13} shows that the same method gives consistent estimates for mixed logit models. This has been explored in an extension of the work presented here in \citep{maelleMixed17}.

Finally, it is worth noting that the expected value function in \refeq{eq:MNL} might pick up some of the correlation in the unobservable $\epsilon$ that is usually captured by introducing nests in a trip or tour based model. Since a trip with walk, public transport and bike all share the same state, except for the arrival time, \eutil will be correlated for the three alternatives. 
