\subsection{future estimation}
Estimation without sampling of alternatives require approximations, for example using sampling of locations. The effect of such approximations can be evaluated on the model proposed in the case study as estimates on that model can be obtained without any approximations. % as was done here with linear in EV approximation

Each individual is considering all possible locations for each new action. Locations are therefore both state variables and alternative actions so the computation time increases quadratically with the number of locations. Restricting the choice set of locations for individuals is therefore extremely tempting, but combining an activity scheduling model with a location choice set model in a consistent way seems extremely complex. This curse of dimensionality connected to the number of zones is sometimes solved by sampling a number of zones through some auxiliary model (see e.g., \citep{liao13}), or by approximating the log-sums through importance sampling (similar to how \citealt{Bradley10} does in a nested framework). We want to avoid such approximations if possible. However, if the zones would be refined or increase for other reasons, we would likely have to resolve to some sort of sampling. \citet{Rust97} shows how randomization can be used to approximate $\eutil$ in dynamic discrete choice models, and it would be one possible way to decrease computation time.

\subsection{To add}

We currently have linear-

\subsection{Comparsion with other models}

One of the 
Compared to \citet{Bowman01}, we include time-space constraints. Can have a finer time diemsion. Choices of duration is consistent with other choices. Potential to include planning under uncertainty. Time-geography.  
Time-space constraints is highlighted in activity based literature, as important when condsidering how people adapt. This together with the improvment in the time-dimension makes it reasonable to assume that the proposed framework could produce better forecast on the effect of policies which influence when people make their trips and how they chain their trips. 
. 

\subsection{Comparsion with recursive logit}
Recursive Logit (RL) models has recently been developed extending the MNL case discussed in \citet{fosgerau2013} to cover Nested Logit \citep{mai2015}, MEV \citep{mai2016method} and Mixed Logit specifications \citep{mai2016decomposition}. They have also been applied in a number of scenarios, e.g., in \cite{zimmermann2017bike}  application to route choice for bikes, possibly with the largest network so far in a RL model. The number of links in these models are between 7\zdel 000-40\zdel 000. The model presented here is thus around $2\zdel 000$ times bigger, so although the models are very similar the estimation techniques used for RL-models are not feasible here. 

\subsection{Correlation between alternatives}
It is common practice in route choice modelling to add a size attribute to each link to take correlation among paths that overlap into account, e.g., using Path-Size Logit \citep{BenAkivaBier99}. For their link-based route choice model, \citet{fosgerau11} obtains a size coefficient by calculating \eutil\, in each link using some pre-specified parameters and adding that to the link-utility. In the activity-scheduling model presented here, it is not as easy to define the overlapping of paths, as the network is dynamic. If two paths are identical besides that the start time for all activities in one path is $10\unit{min}$ after the start time in the other path, there can be practically no overlapping as defined by the Path-Size Logit although the two paths would be very similar. How to address this issue in a dynamic network and in the activity-scheduling framework is therefore an open question.

In trip-generation models, it is common to have nests for mode choice, location choice and activity choice, as in, e.g., \citet{Bowman01}. It would be possible to introduce different scales for the error term when solving \refeq{eq:EV} and obtaining choice probabilities in \refeq{eq:MNL} where the scale (which is one here) would be state dependent. This is done in the Nested Recursive Logit model described in \citet{mai2015}. However, the probability of a path would then not reduce to  \refeq{eq:pPath} and sampling of alternative sequences would not be possible to use for estimation.
\citet{guevara2013MEV} recently showed that Multivariate Extreme Value (MEV) models such as Nested Logit can be estimated using sampling of alternatives. A Nested RL-model is however not the same as an MEV model, so the transferability of the result is uncertain. If nests are introduced within the network, as in \citet{mai2015}, it would further require that the value function was approximated in all states, so the computational benefit might not be enough. An alternative would be to introduce nests over paths, for example nesting alternatives that include specific activities or modes. This would move the model in the direction of \citet{Bowman01}. The computation time is however likely to grow linearly with the number of nests. Given that the model already is time demanding, creating nests for all combinations of modes and activities would not be computationally feasible.

 Another issue is the correlation in preferences over time. Individuals' variances in preferences for, e.g., mode or activities are likely to be consistent over time and therefore to some extent be the same throughout the day. Including nests on a trip level would not capture this correlation. A possible solution would be to introduce mixed parameters for, e.g., activities and modes, that would be the same for each individual for the full day. Our estimation approach is based on sampling of alternatives and recent research by \citet{Guevara13} shows that the same method gives consistent estimates for mixed logit models. This has been explored in an extension of the work presented here in \citep{maelleMixed17}.

Finally, it is worth noting that the expected value function in \refeq{eq:MNL} might pick up some of the correlation in the unobservable $\epsilon$ that is usually captured by introducing nests in a trip or tour based model. Since a trip with walk, public transport and bike all share the same state, except for the arrival time, \eutil will be correlated for the three alternatives.