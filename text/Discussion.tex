


\subsection{Comparison with nested logit model}
As the DDCM travel demand model presented in this paper closely resembles the nested-logit approach used in, e.g, \citet{bowman2001}, we think it is worth comparing the two modelling approaches in some more detail. Firstly, the DDCM model is formulated in a way that allows for modelling of planning under uncertainty. It thus allows the incorporation of random events during the course of the day that individuals may react to, especially related to traffic conditions and travel time. Secondly, the natural inclusion of time in the DDCM model also seems to indicate that it is possible to have a finer discretization of the time dimension, and in the case study of Section 4 we discretized time into 10-minute intervals. Principally, it is possible to have a very fine discretization of the time dimension in Bowman Ben-Akiva type models, but in practice the choice dimension has been limited for computational reasons. For comparison, \citet{bradley2010sacsim} discretized time into 30-minute intervals and only allowed a sampled subset of these discretized time periods to actually be chosen for each individual. Further, in \citet{bradley2010sacsim}, individuals choose at which of the sampled time steps to end each activity, rather than the duration of each activity as in the DDCM approach in this paper, made possible by the fact that time is seen as continuous. The possible duration's of an activity will thus depend on the origin and destination of the trip leading to the destination. For example, if the duration of a trip would be 29 minute, the duration of an activity can be 1 minute, 31 minutes, etc., whereas if the trip is 31 minutes the duration can be 29, 59, etc. That may have an unintended effect on the joint destination-duration choice. All in all, we believe that the finer discretization of the time dimension and the potential to include planning under uncertainty in the DDCM makes it plausible that will perform better when used to predict the outcome of policies which influences timing of trips and especially the reliability of different transport modes. 

\subsection{Correlation between alternatives}
It is common practice in route choice modelling to add a size attribute to each link to take correlation among paths that overlap into account, e.g., using Path-Size Logit \citep{BenAkivaBier99}. For their link-based route choice model, \citet{fosgerau2013} obtains a size coefficient by from an approximation of the expected flow on each link obtained some non-estimated parameters. Calculating such a size-attribute of the type proposed in \citet{fosgerau2013} would be possible in the DDCM models discussed here. However, it is not as easy to define overlapping of paths in the dynamic activity-travel network presented here as in the car network as the network is dynamic. If two paths are identical besides that the start time for all activities in one path is $10\unit{min}$ after the start time in the other path, there can be practically no overlapping as defined by the Path-Size Logit although the two paths would be very similar. 

In trip-generation models, it is common to have nests for mode choice, location choice and activity choice, as in, e.g., \citet{bowman2001}. It is worth noting that the expected value function in \refeq{eq:MNL} might pick up some of the correlation in the unobserveable $\epsilon$ that is usually captured by introducing nests in a trip or tour based model. Since a trip with walk, public transport and bike all share the same state, except for the arrival time, \eutil will be correlated for the three alternatives. It would be possible to change the assumption on the error term to allow for nests in each choice situation. This is done in the Nested Recursive Logit model described in \citet{mai2015}.
\citet{guevara2013MEV} recently showed that Multivariate Extreme Value (MEV) models such as Nested Logit can be estimated using sampling of alternatives. A Nested RL-model is however not the same as an MEV model, so the transferability of the result is uncertain. If nests are introduced within the network, as in \citet{mai2015}, it would further require that the value function was approximated in all states. An alternative would be to introduce nests over paths, for example nesting alternatives that include specific activities or modes. This would move the model in the direction of \citet{bowman2001}. The computation time is however likely to grow linearly with the number of nests. Given that the model already is time demanding, creating nests for all combinations of modes and activities would not be computationally feasible.

Recursive Logit models have recently been developed extending the MNL case discussed in \citet{fosgerau2013} to cover Nested Logit \citep{mai2015}, MEV \citep{mai2016method} and Mixed Logit specifications \citep{mai2016decomposition}. They have also been applied in a number of scenarios, e.g., in \cite{zimmermann2017bike}  application to route choice for bikes, possibly with the largest network so far in a RL model. The number of links in these models are between 7\zdel 000-40\zdel 000. From the discussion in Section \ref{seq:computationTime} it is clear that the model presented here is around $2\zdel 000$ times bigger, so although the models are very similar the estimation techniques used for RL-models cannot be directly transferred to this context. 

 Another issue is the correlation in preferences over time. Individuals' variances in preferences for, e.g., mode or activities are likely to be consistent over time and therefore to some extent be the same throughout the day. Including nests on a trip level would not capture this correlation. A possible solution would be to introduce mixed parameters for, e.g., activities and modes, that would be the same for each individual for the full day. Our estimation approach is based on sampling of alternatives and recent research by \citet{guevara2013mixed} shows that the same method gives consistent estimates for mixed logit models. This has been explored in an extension of the work presented here in \citep{maelleMixed17}.
 
 \subsection{Future work: sampling of locations}
In the case study presented in this paper we estimated the DDCM model using sampling of alternative daily travel patterns. Doing so allows for consistent estimates only in the special cases when either all error terms area i.i.d. Gumbel or potentially when they are distributed to create nests over sets of full daily travel patterns \citep[using][]{guevara2013MEV}. Sampling of sequences will not work in a model including travel time uncertainty or different distributions of the error structure witihn the day, and estimating the model for such cases will therefore be extremely time consuming (see discussion in Section \ref{seq:computationTime}). By using sampling of alternative sequences it is possible to estimate the model with very few restrictions on the full choice set, and in the case study the model was estimated with 1240 alternative possible locations for each trip. Each individual is considering all possible locations for each new action. Locations are therefore both state variables and alternative actions so the computation time increases quadratically with the number of locations. The curse of dimensionality connected to the number of zones is sometimes solved by sampling a number of zones through some auxiliary model (see e.g., \citep{liao2013incorporating}), or by approximating the log-sums through importance sampling (similar to how \citealt{bradley2010sacsim} does in a nested framework). \citet{Rust97} shows how randomisation can be used to approximate $\eutil$ in dynamic discrete choice models, and it would be one possible way to decrease computation time.
In order to speed up the model in the future, we believe approximating the model by sampling locations may be necessary both for simulation and estimation. Work in this direction has been started in \citet{saleem18largeScale}, where sampling of alternatives was used for simulation in order to allow for the model to feed demand to a micro-simulator. One great benefit of being able to estimate and simulate the large scale model presented in this case study is that it allows the effect of any approximations to be evaluated. 
It may be possible to approximate using reinforcement learning. \citep{vanhusel09} showed the feasbility when only scheduling of activities was considered. It may be possible to simulate from the mnl version using metropolis hastings \citep{danalet}.% as was done here with linear in EV approximation