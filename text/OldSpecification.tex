Defining the choice set $C(x_k)$ involves defining the set of purposes, destinations and modes that are available in a specific state $x_k$.
In principle, an individual can in each time step decide between: either staying at the same location and perform the previous activity $p$ for a while longer; or travel to another location and start a new activity there. In the first case, we assume that each activity purpose $p$ has a minimum additional duration $\Delta_p t$.
However, the availability of activities are limited in time and space. Due to such time-space constraints it may not be possible to continue with an activity after a certain time, and it is in general not possible to start any activity at any location. For example, home and work activities are restricted in space to the home and work location, and the work activity is potentially restricted in both duration and time-of-day by the individuals working schedule. One can view such time-space restrictions as location specific opening hours for each activity: an activity with purpose $p$ can only be performed at location $l$ between the opening time $t^{\text{open}}({l,p})$ and closing time $t^{\text{close}}({l,p})$. Certain activities, e.g., work, may also have a minimum and maximum duration for each episode. The minimum duration for a purpose $p$ will be denoted $\tau^{min}(p) $ and the maximum duration $\tau^{max}(p)$. 

If $p=p_\text{travel}$, it means that the agent travelled to the current location with the previous action with the purpose to start a new activity in the current decision stage. The mode and destinations choice is therefore restricted to $\act{m} = m_{\text{stay}}$ and $\act{d}=l$. The set of available purposes is determined by the time space constraints as:
$P(t,l) = \big \{ p \in P_{act} \;\text{s.t:} \; t^{\text{open}}({l,p}) \leq t \leq t^{\text{close}}({l,p}) - \tau^{min}(p) \big \} $, i.e., it must be possible to start the activity and pursue the activity for its minimum duration $\tau^{min}(p)$ within the opening hours at the specific location.



If $p=p_\text{end}$, the agent has ended an activity with the previous action in order to travel in the current decision stage, so $\act{p}=p_\text{travel}$. The available alternatives thus consist of any combination of destination $\act{d} \in L$ and mode $\act{m}\in M(m)\subset M$ where the available modes $M(m)$ is assumed to only depend on the state variable $m$. 


If $p\in P_{act}$, the agent is currently performing an activity and can either continue with the same activity of another time step or end the activity and travel in the upcoming decision stage. It is possible to continue with an activity as constraints on maximum duration and opening hours are satisfied, i.e., if $t+\Delta_p t \leq t^{\text{close}}({l,p})$ and $\tau_p + \Delta_p t \leq \tau^{max}(p)$

\begin{align}
    C(x_k) = \left \{
    {\def\arraystretch{2}\tabcolsep=2pt
    \begin{tabular}{rrrrccl}
  \Big\{ & $l$, &$m_\text{stay}$,& $P(t,l)$         &\Big\}  &if&  $p=p_\text{travel}$\\
  \Big\{ & $L$, &$M(m)$,         &$p_\text{travel}$   &\Big\}  &if&$p = p_{\text{end}}$ \\
  \Big\{ & $l$, &$m_\text{stay}$,&$\{p,p_\text{end}\}$&\Big\}  &if& $p \in P_{act}$ \text{and} $t+\Delta_p t \leq t^{\text{close}}({l,p})$ \\
  \Big\{ & $l$, &$m_\text{stay}$,&$p_\text{end}$&\Big\}  &if& $p \in P_{act}$ \text{and} $t+\Delta_p t > t^{\text{close}}({l,p})$ \\
  \end{tabular}
  }
    \right.
\end{align}

The individual is able to stay in the current location and continue with the current activity for an activity dependent duration $\Delta_{p}t$. We model the choice of the total duration of an activity as a sequence of decisions to continue with the same activity. The total duration of an activity is thus always a multiple of $\Delta_{p}t$. In terms of the action space, this is modelled by adding a 'Stay' mode, only available in combination with $\act{d}=l$ and $\act{p}=p$, i.e., when the destination equals the origin and the purpose equals the ongoing activity.