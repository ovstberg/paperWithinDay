%1 : Why ABM?
Travel demand models have during the last decades evolved from highly aggregated trip based models, through tour based models into activity based models that considers the choice of full daily activity-travel patterns including all activities and all transportation at an individual or household level. For an overview of activity based modelling up to date, the readers are referred to the reviews by \citet{Pinjari11} or \citet{Rasouli14}.

To achieve an interdependent choice a full daily travel pattern, it is common to assume that individuals (or households) make the decision of how to plan their day based on some sort of scoring or utility function including the utility obtained from all activity episodes and the (dis)utility from travelling. Possible the first model based on this assumption was presented by \citet{Adler79}, who assumed that individuals behaved as utility and thus choose the utility maximising travel pattern $tp$ among the set $TP$ of all feasible travel patterns:
\begin{equation}
\label{eq:opt}
	\begin{aligned}
	& \underset{tp}{\text{maximize}}
	& & U_{tp} \\
	& \text{subject to}
	& & tp \in TP.
\end{aligned}
\end{equation}
To actually implement a model based on this assumption in 1979 they had to impose some additional restrictions: firstly the choice set $TP$ was constructed from all observed travel patterns; and secondly timing of trips were not modelled. One of the great challenges with implementing a model for the choice of a full travel pattern is indeed how to formulate the universal choice set, which can be enormous. Some models has avoided the need to formulate the universal choice set by formulating a random process by which alternative travel patterns are derived, and compare a smaller set of alternative travel patterns obtained from this process based on their full utilities. This was the method used in Starchild \citet{recker86starchild1,recker86starchild2} and is the method used by MATSim for route \citep{lefebvre2007fast}, mode \citep{grether2009mode}, timing \citep{balmer05} and location \citep{horni2011} of all trips. Other models are based on the idéa that individuals in the event of changes to environment adapt their travel patterns using different techniques and evaluate whether the perturbations offer any improvement in terms of full daily utility. This was the idéa behind AMOS \citep{kitamura1996Sams}. Somewhat similarly, AURORA focuses on how individuals reschedule a given set of activities by adjusting their travel pattern until further adjustments does not motivate the search cost \citep{timmermans2001modeling,joh2003Aurora, johEstimationAurora2005}.


All of the above discussed models assume that individuals consider the full utility of a day when planning their travelling, but neither provide solutions for how to find the actually optimal travel pattern. Some models have been proposed which overcomes this limitation. In the Household Activity Pattern Problem (HAPP), the choice of all household members daily travel patterns including time-of-day, mode and location choices is formulated as constrained mixed-integer optimisation problem \citep{Recker01,Recker08,Recker13,yuan2014HAPP}. The problem is solvable, but computationally very demanding and with only 19 locations the computation time is reported to an average of $614\units{s/household}$ \citep{Recker13}.

The choice of a travel pattern can also be formulated as a shortest-path problem in a an abstract network, as in the multistate-supernetwork model \citep{arentze04Multistate}. In such a network, states can be defined for remaining mandatory activites and thus capture time-space constraints \citep{liao2013incorporating} \citep{liao2016modeling}. Similarly to HAPP; the computation time is still a limiting factor. \citet{liao2016modeling} reports that the evaluating the choice of a daily travel pattern takes $8\units{s}$ when an individual has two fixed activities to perform in a day and 6 locations to choose from. It is further not yet clear how to estimate the model, as parameters currently are obtained by using separate MNL models for different choice dimensions in \citet{Liao2017}. The model estimated is thus not identical  with the shortest path algorithm used for model simulation in, e.g., \citet{liao2016modeling}.

An alternative to consider the choice of a full daily travel pattern at once has been to model a sequential choice of, e.g., whether or not to perform a trip, departure time, mode of transport, destination and purpose. The probability to make a new choice then modelled conditional on previous choices, and a sequence of choices will produce a daily travel pattern. This is the approach taken in, e.g., Albatross \citet{timmermans2001modeling} and CEMDAP \citep{bhat04}. This means that the full history can influence the probability of a choice. It further means that the models used in each choice situation can be very complex. However, the choice probabilities in the above mentioned models are insensitive to changes that occur in the future.  

and that  This means that, e.g., the decision off the departure time to work in the morning will be independent of changes to the traffic network affecting the return trip in the afternoon. 


- Many models proposed which considers all activity episodes, and all trips performed during a day. 
- One focus of activity based model is to obtain an interdependent choice of all aspects. Has been argue by many. 
- In order to obtain an interdependent choice, it has been common to assume that individuals or households considers the utility of a full day. Some models 
- liao, happ, matsim also route, most important.. Motivates the search for new methods. 
- Bowman. Difficult to include time-dependence between tours.

- \citet{timmermans2001modeling} base the model on Albatross on the assumption that individuals form condition-action pairs 
- Habib
- reenforcment learning + markov chains

- Further, non of the above mentioned frameworks has managed to include deicsion making under uncertain travel times. Attempts to model rescheduling within a travel demand model instead often focus on how to adapt the remaining travel pattern once a delay has been experienced. This is a certainly valid when considering unexpected delays, but seems inaccurate when describing how individuals plan for expected delays from, e.g., daily fluctuations in travel time due to traffic congestion. Planning under uncertainty is instead modelled by scheduling models. 



A natural way to accomplish an interdependent choice of all aspects of a daily travelling would be to directly model the choice of a full daily travel pattern.
The problem with this approach is that the number of possible ways to plan a day is immense. One could definitely argue that people do not actually consider all these alternatives, but there are definitely complex aspects of the activity schedule that do influence how people plan their days. For example, when considering what time to leave for work, they are likely to take into consideration when they will get home and so the preferred departure time to work should be derived from a trade-off between time spent at home in the morning versus time spent at home (or on some activity) in the evening.

%Since all aspects of a travel pattern is interconnected, it seems reasonable to assume that individuals evaluate the full pattern when determining what activities to engage in as well as timing, location and mode of transport. Assuming that individuals (or households) act as if the consider the utility of a full daily travel pattern is not a new idea. It was proposed in \citet{Adler79}, who base a model on the assumption that households make a joint choice of a travel pattern $tp$, considering travel time, mode and destination choices for all tours during a day by selecting the optimal pattern $tp^*$ from the set of feasible patterns $TP$ according to: $tp^* = \argmax_{tp \in TP} U(tp) + \epsilon_{tp}$. With $\epsilon_{tp}$ Gumble distributed, this becomes an MNL model which was estimated using the observed sample as a choice set. Individuals are also considering the full daily travel pattern in: \citet{Recker08}, who formulates the model as a mathematical programming problem; and in MATSim \citep{balmer05,horni2011} who uses simulation to obtain close too optimal solutions. In AURORA \citep{joh2003Aurora,johEstimationAurora2005}, individuals are considering the utility of the full daily travel pattern but are assumed to use search-heuristics to schedule (and reschedule) their day. 
%%2 : ABM so far 
How to spend the limited time budget when the preferences for time is dependent on the time of day should be the key determinant for when, where and if people choose to conduct different activities in activity based models. Although many activity based models up to date result in full day activity schedules, most fall short in their treatment of time. For a comprehensive overview of activity based models, see e.g., \citet{Pinjari11} or \citet{Rasouli14}. As an extension to the tour-based approach, \citet{Bowman01} developed a nested-logit structure that treats tours and activities sequentially based on their importance for the individual. The model consists of five nests: 1) the choice of activity pattern, including the number of tours carried out during the day; 2) the choice of time of day for the primary tour and all its trips; 3) the mode and destination for the primary tour; 4) the time of day for the secondary tours; and 5) the mode and destination of the secondary tours. The nested-logit structure ensures that higher decisions, such as the choice of activity pattern, includes the individual specific information about all available tour-combinations that the pattern includes. However, since secondary tours are not conditioned on each other, they are not temporarily consistent. For instance, it is technically possible to end up with daily patterns that take more than a full day to complete. The model has been further developed by combining with a duration and departure time model, which is not integrated into the nested-logit structure \citep{Vovsha04,Bradley10} and so does not integrate upwards. 
% The probability of choosing a tour that ends at 5 PM should depend on the expected utility of the rest of a day when being home at 5 PM. If one more tour is to be conducted during the remaining time of the day, then they can be started at 10 alternative times before 10 PM. For the secondary tour, taking place after 5 PM, the expected utility should in turn depend on the starting time, since that will influence the number of available destinations that can be reached, as well as the time that can be spent on different activities. It should also depend on when one is finished with all activities.

There are a number of models in the literature that is related to the approach taken in this paper. First, \citet{Habib11RUM} presents a discrete-continuous random utility model for weekend travelling. Agents choose mode, destination and activity based on the utility of the combination. Future time is contained in a time-of-day dependent composite good, which is parameterized and estimated. Second, in the Albatross model system, choices are also made sequentially in time \citep{Arentze00}, and at every time step a heuristic decision rule determines determine the next action so that time-space constraints are fulfilled. One challenge with these approaches is the difficulty in treating value of future time consistently, in the sense that a dynamically consistent model should either directly model the choice of full day-schedules, or the value of future time that individuals consider should be the same as the expectation of utility that they can obtain during the remaining day.
%and \citep{bhat04} 

%3 : Problems with time + rescheduling
%One way to consistently treat time is to base the choice partly on how the time is spent during a day. If the time always sums up to a fixed amount and all alternative schedules fulfill time-space constraints, the model will be time-consistent. One way of representing this problem is as the choice of a path in a dynamic network. 
%
% Models that try to overcome try to models the choice as day-paths or activity chains have therefore focused on alternative methods. One intuitive way is to represent the 
%possible that this shouldn't be here. 
%One approach is to combine activities through a dynamic network of the transportation network. A path in this network would constitute a daily activity schedule, and can include choice of destination, mode, activity, route and parking position \citep{Arentze04,liao13}.  The framework have been extended to include time-space constraints through constraints on the feasible paths \citet{liao13}. 

%5 : Problem with these models
The frameworks mentioned above are promising in their attempts to model and simulate the choice of daily travel patterns. 
In this paper, we follow the random utility-based approach where individuals maximize the achieved utility from a path through activity-location-time-space throughout one day. Following this approach, we acknowledge that it is unrealistic that a single individual consider all possible paths during one day. The rationale following this utility-maximization approach is twofold. First, the standard practice in the field is based on random utility maximization, and even more specifically nested logit. The approach taken in this paper closely follows the standard practice, but introduces time explicitly, respecting that time has a direction and that decisions in real-life actually can be made sequentially in time, given the information available at each point in time when a decision is made. Second, individual rationality is to-date a corner stone for welfare economics, enabling the approach to be theoretically and consistently translated into a tool for cost-benefit analysis. Both of these arguments are strong, independently, and the latter argument is difficult to ignore without having a theory for behavioural welfare economics in sight. 
% zxc(CITATION NEEDED).  

Further, although they are based on utility maximization, it is unclear how they should be used for other purposes than prediction. In a discrete choice framework, it is logical to use the expected (maximum) utility from a day-path as input to other models and for cost appraisal \citep{Geurs10}. The expected utility could also be used to get detailed disaggregated measures of accessibility \citep{Dong06, Jonsson13}. With a dynamically consistent model, such measures of accessibility could be used to see how, e.g., the fact that some activities such as picking up children and going shopping are mandatory influences the accessibility of different work locations.

%Start with the three similar approaches, then go on to older activity based where not consistent Bhat, albatros, bowman

The number of different daily activity patterns is immense, but the number of possible actions available to an individual at a specific time of the day is relatively easy to define. Dynamic discrete choice theory could therefore present a way of simplifying the activity scheduling problem without making any restrictions on the choice set, as proposed by \citet{karlstrom04}. It could also give detailed time-dependent individual accessibility measures, as demonstrated by \citet{Jonsson13}. The basic idea is to model choice of a travel patterns as a sequence of simultaneous choices of activity type, duration, mode of transport and location conditional on the expected future utility given respectively choice. The sequencing of actions and the expected future utility components ensures that both the history and the future is taken into account; makes it easy to include time-space constraints; and allows for rescheduling due to unexpected events. However, the idea has only been implemented in small example cases and no model has been previously estimated. 

In this paper we propose an estimation method for a dynamic discrete choice activity based model based on sampling of alternatives that is used to estimate an implementation of the model. Section 2 will present and discuss the modelling framework and specification; section 3 will discuss the estimation method proposed; section 4 the data; section 5 the utility specification; and section 6 estimation result and simulation validations.
