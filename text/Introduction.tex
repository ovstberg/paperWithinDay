%1 : Why ABM?


Travel demand models have during the last decades evolved from highly aggregated trip based models, through tour based models into activity based models that considers the choice of full daily activity-travel patterns including all activities and all transportation at an individual or household level. One motivation behind this gradual increase in complexity is the idea that the demand for travel is derived from the demand for activity participation. The demand for activities should therefore be the determinant for if, when and where trips are being performed. There is also a focus on the interdependence of all choices performed during the day as trips are naturally linked an space-time. To capture this interdependence, all facets of daily activity-travel planning should be modelled interdependently. Accurate predictions of how individuals will react to infrastructure investments and policy changes also need to take into account to what extent individuals are flexible in their activity participation. There is only a limited amount of time each day and some activities are more or less fixed in both location and time, such as working and picking up or dropping of children at school/daycare. Such constraints severely restrict individuals' ability to adapt to changes in the transportation system, and not including such constraints in models used for forecast may produce unrealistic result. This is especially the case when considering policy changes such as congestion charge that are becoming increasingly important in today's traffic planning. 

To achieve an interdependent choice a full daily travel pattern, it is common to assume that individuals (or households) make the decision of how to plan their day based on some sort of scoring or utility function including all activity and travel episodes of the day. Possible the first model based on this assumption was presented by \citet{Adler79}, who assumed that individuals behaved as if they choose the utility maximising travel pattern among the set of all feasible travel patterns.
To actually implement a model based on this assumption in 1979 they had to impose some additional restrictions: firstly the choice set was constructed from all observed travel patterns; and secondly timing of trips were not modelled. Indeed, one of the great challenges with implementing a model for the choice of a full activity-travel pattern is still the immense size of the choice set. Alternatives methods that does not require choice set formation has therefore been developed. In the Household Activity Pattern Problem (HAPP), the choice of all household members daily travel patterns including time-of-day, mode and location choices is formulated as constrained mixed-integer optimisation problem \citep{Recker01,Recker08,Recker13,yuan2014HAPP}. The problem is solvable, but computationally very demanding and with only 19 locations the computation time is reported to an average of $614\units{s/household}$ \citep{Recker13}. The choice of an activity-travel pattern can also be formulated as a shortest-path problem in a an abstract network, as in the multistate-supernetwork model \citep{arentze04Multistate}. In such a network, states can be defined for remaining mandatory activites and thus capture time-space constraints \citep{liao2013incorporating} \citep{liao2016modeling}. Similarly to HAPP; the computation time is still a limiting factor. \citet{liao2016modeling} reports that the evaluating the choice of a daily travel pattern takes $8\units{s}$ when an individual has two fixed activities to perform in a day and 6 locations to choose from. It is further not yet clear how to estimate the model, as parameters currently are obtained by using separate MNL models for different choice dimensions in \citet{Liao2017}. The model estimated is thus not identical with the shortest path problem solved when the model is used for simulation in, e.g., \citet{liao2016modeling}.

Some models have avoided the need to formulate the universal choice set by constructing a random process by which alternative travel patterns are derived, and compare a smaller set of alternative travel patterns obtained from this process based on their full utilities.  
This was the method used in Starchild \citep{recker86starchild1,recker86starchild2}. A similar approach is also used in MATSim which besides mode, timing and location also include route choice for of all trips although the number of activities and their ordering is fixed  \citep{lefebvre2007fast,grether2009mode,balmer05,horni2011}. Some methods have been proposed that set out to mimic how individuals adapt their travel patterns in the event of changes to environment and evaluate whether the perturbations offer any improvement in terms of full daily utility. This was the idea behind AMOS \citep{kitamura1996Sams}. Somewhat similarly, AURORA focuses on how individuals reschedule a given set of activities by adjusting their activity-travel pattern until further adjustments does not motivate the search cost \citep{timmermans2001modeling,joh2003Aurora, johEstimationAurora2005}.

An alternative to consider the choice of a full daily travel pattern as a joint choice has been to model a sequential choice of, e.g., whether or not to perform a trip, departure time, mode of transport, destination and purpose. The probability to make a new choice is then modelled conditional on previous choices, and a sequence of choices will produce a daily travel pattern. This is the approach taken in, e.g., Albatross \citep{timmermans2001modeling} and CEMDAP \citep{bhat04}. In such sequential models, the full history can influence the probability in any choice situation, ensuring an interdependence between trips. \citet{Habib11RUM} developed a discrete-continuous random utility model for weekend travelling where agents decided on mode, destination and activity based on the utility of the combination. Agents in the model of \citet{Habib11RUM} also considered how their decisions influenced future opportunities by including a utility component for the value of future time. The value of future time was modelled as a time-of-day dependent composite good, which was parameterized and estimated. 

%Since all aspects of a travel pattern is interconnected, it seems reasonable to assume that individuals evaluate the full pattern when determining what activities to engage in as well as timing, location and mode of transport. Assuming that individuals (or households) act as if the consider the utility of a full daily travel pattern is not a new idea. It was proposed in \citet{Adler79}, who base a model on the assumption that households make a joint choice of a travel pattern $tp$, considering travel time, mode and destination choices for all tours during a day by selecting the optimal pattern $tp^*$ from the set of feasible patterns $TP$ according to: $tp^* = \argmax_{tp \in TP} U(tp) + \epsilon_{tp}$. With $\epsilon_{tp}$ Gumble distributed, this becomes an MNL model which was estimated using the observed sample as a choice set. Individuals are also considering the full daily travel pattern in: \citet{Recker08}, who formulates the model as a mathematical programming problem; and in MATSim \citep{balmer05,horni2011} who uses simulation to obtain close too optimal solutions. In AURORA \citep{joh2003Aurora,johEstimationAurora2005}, individuals are considering the utility of the full daily travel pattern but are assumed to use search-heuristics to schedule (and reschedule) their day. 
%%2 : ABM so far 

% For a comprehensive overview of activity based models, see e.g., \citet{Pinjari11} or \citet{Rasouli14}. 


In practice, random utility maximization models and specifically Nested Logit models are commonly used for travel demand analysis. One of the most extensive such models was presented by \citet{Bowman01} where a nested-logit structure is developed for the choice of all trips performed during a day. The model treats tours and activities sequentially based on their importance for the individual. The model consists of five nests: 1) the choice of activity pattern, including the number of tours carried out during the day; 2) the choice of time of day for the primary tour and all its trips; 3) the mode and destination for the primary tour; 4) the time of day for the secondary tours; and 5) the mode and destination of the secondary tours. The nested-logit structure ensures that higher decisions, such as the choice of activity pattern, includes the individual specific information about all available tour-combinations that the pattern includes. Many extensions to the original prototype model presented in \citet{Bowman01} has since been performed, to, e.g., allow for a greater detail in time-of-day choice \citep{Vovsha04}. In a recent implementation, time-of-day is modelled at a 30 minute resolution, where each individual consider a sampled subset of possible 30-minute intervals for each trip \citep{Bradley10}.
% The probability of choosing a tour that ends at 5 PM should depend on the expected utility of the rest of a day when being home at 5 PM. If one more tour is to be conducted during the remaining time of the day, then they can be started at 10 alternative times before 10 PM. For the secondary tour, taking place after 5 PM, the expected utility should in turn depend on the starting time, since that will influence the number of available destinations that can be reached, as well as the time that can be spent on different activities. It should also depend on when one is finished with all activities.

%and \citep{bhat04} 

In this paper, we propose to use a Markov Decision Process (MDP) to model the choice of a daily activity-travel pattern. 
Choices are made sequentially in time, but we follow the random utility-based approach where individuals are assumed to maximize the expected achieved utility from a path through activity-location-time-space throughout one day. Formulating the MDP as a dynamic discrete choice model (DDCM), it is possible to derive choice probabilities so that maximum likelihood estimation can be performed \citep{RustML88}.
The rationale following this utility-maximization approach is twofold. First, the standard practice in the field is based on random utility maximization, and even more specifically nested logit. The approach taken in this paper closely follows the standard practice, but introduces time explicitly, respecting that time has a direction and that decisions in real-life actually can be made sequentially in time, given the information available at each point in time when a decision is made. Second, individual rationality is to-date a corner stone for welfare economics, enabling the approach to be theoretically and consistently translated into a tool for cost-benefit analysis. Both of these arguments are strong, independently, and the latter argument is difficult to ignore without having a theory for behavioural welfare economics in sight. 

In a discrete choice framework, it is logical to use the expected (maximum) utility from a day-path as input to other models and for cost appraisal \citep{Geurs10}. The expected utility could also be used to get detailed disaggregated measures of accessibility \citep{Dong06, Jonsson13}.

As choices are made sequential in time, so as in Cemdap and albatross there is no need to formulate the universal choice set. Similarly to the sequential choice model of \citep{Habib11RUM}, individual considers a future utility component in each choice situation, but rather than parameterizing the value of the future, the proposed model explicitly calculates the expected future remaining utility conditional on the choice. This ensures that individuals act as if they considers expected utiltiy of a daily activity-travel pattern, similarly to the underlying assumption of, e.g., MATSim, Happ and the multistate-supernetwork model discussed above. However, compared to these models, we see as a benefit of our approach that choice probabilities and welfare measures can be obtained.
Similarly to \citep{Bowman01} type models, the proposed model can be used to produce welfare measures using the log-sum. In a discrete choice framework, it is logical to use the expected (maximum) utility from a day-path as input to higher order models and for cost appraisal \citep{Geurs10}. The expected utility could also be used to get detailed disaggregated measures of accessibility \citep{Dong06}.

It could also give detailed time-dependent individual accessibility measures, as demonstrated by \citet{Jonsson13}. The basic idea is to model choice of a travel patterns as a sequence of simultaneous choices of activity type, duration, mode of transport and location conditional on the expected future utility given respectively choice. The sequencing of actions and the expected future utility components ensures that both the history and the future is taken into account; makes it easy to include time-space constraints; and allows for rescheduling due to unexpected events. However, the idea has only been implemented in small example cases and no model has been previously estimated. 

In this paper we propose an estimation method for a dynamic discrete choice activity based model based on sampling of alternatives that is used to estimate an implementation of the model. Section 2 will present and discuss the modelling framework and specification; section 3 will discuss the estimation method proposed; section 4 the data; section 5 the utility specification; and section 6 estimation result and simulation validations.
