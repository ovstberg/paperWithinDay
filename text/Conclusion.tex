During the last decades, many activity-based models have been developed in the literature. However, there is still a lack of random-utility based models for which time integrates consistently in all choice dimensions. A natural approach would be to introduce time explicitly in the models, respecting that time has a direction and that it is possible to make decisions sequentially taking into account the available information at that time. It is also natural to respect that people are not completely myopic, but are capable of forward--looking, for instance taking into account the consequences for afternoon activity opportunities when deciding whether to take the car to work in the morning. 

The challenge with such a natural extension of the existing state-of-practice modelling framework is the immense combinatorial problem of, at least technically and consistently, considering all possible combinations of activity-location pattern throughout one single day, let alone combinations of days. In this paper, we formulate a dynamic discrete choice model which overcomes this curse of dimensionality using dynamic programming. In the framework, time is respected in the above-mentioned aspects making it dynamically consistent. 
We also demonstrate that it is indeed possible to estimate the model. Estimation is the main purpose and the main achievement of this paper. The proposed and thus estimated model is also validated in-sample. 

There are a number of immediate extensions of this model that can and will be explored in further research. First, it is natural to extend the model to multiple days. Some activities can be postponed to later days, and there is interaction between activity patterns during consecutive days. For instance, shopping for food is an activity in which a planning horizon of more than one day is very relevant to consider. 

Another limitation of the model proposed in this paper is the IID assumption between daily activity schedules. In a sequential decision context, it may be important to consider fixed effects, in particular recognizing that the same individual is making the decisions throughout one day. A natural extension of this model is therefore to consider a mixed panel logit model, where preferences for, e.g., cost, time, modes and activities are heterogeneous between individuals but constant throughout the day for a single individual.

The main challenge addressed in this paper was consistent estimation. This has hindered the use of such models in the past. Another very important aspect is to operationalize the model in implementation. For instance, when using the proposed model in the context of (or in conjunction with) a Dynamic Traffic Assignment (DTA) model, it will be necessary to repeatedly simulate travel schedules for millions of individuals. The naive approach would be to first calculate the value function in all states for each individual, but with the current specifications this would be prohibitively time consuming. Methods to speed up this simulation must therefore be developed.